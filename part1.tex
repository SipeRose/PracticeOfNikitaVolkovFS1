% ---------------------------- Problem 1----------------------------------
\subsubsection*{\center Задача № 1.}
{\bf Условие.~}
Разложить в ряд Фурье заданную функцию $f(x)$, построить графики $f(x)$ и суммы ее ряда Фурье. Если не указывается, какой вид разложения в ряд необходимо представить, то требуетчя разложить функцию либо в общий тригонометрический ряд Фурье, либо следует выбрать оптимальный вид разложения в зависимости от данной функции.

\[
f(x)=\begin{cases}
	0,		&-2 \leqslant x<-1, 	\\ 
	1,		&-1 \leqslant x<1, 	\\ 
	x^{2}, 	& 1 \leqslant x \leqslant 2.
\end{cases}
\]
{\bf Решение.~}	
%График
\begin{center}
	\begin{tikzpicture}
		\begin{axis}[xmin=-3,	xmax=3, 	ymin=-0.5,	ymax=4,
			width=1.1\textwidth,
			height=0.8\textwidth,
			axis x line=middle,
			axis y line=middle, 
			every axis x label/.style={at={(current axis.right of origin)},anchor=west},
			every inner x axis line/.append style={|-latex'},
			every inner y axis line/.append style={|-latex'},
			minor tick num=1,			
			axis equal=true,
			xlabel=$x$, 
			ylabel=$y$,          
			samples=100,
			clip=true,
			]
			\addplot[color=blue, line width=1.5pt,domain=-2:-1] {0};
			\addplot[color=blue, line width=1.5pt,domain=-1:1]{1};
			\addplot[color=blue, line width=1.5pt,domain=1:2]{\x^2};
			\addplot[thick,dashed] coordinates {(1,0) (1,1)};
			\addplot[
			mark=*,
			mark options={fill=black, draw=black},
			only marks,
			] coordinates {(1, 1)};
		\end{axis}
	\end{tikzpicture}
\end{center}
\noindent
Построим общий тригонометрический ряд Фурье вида
$$
f(x)=\frac{a_0}{2}+\sum_{n=1}^\infty 
\left(a_n\cos{(n\omega x)}+b_n\sin{(n\omega x)}\right),\quad\text{где}\,\omega=\frac{2\pi}{T},\,T=4.
$$
\noindent
Вычислим коэффициенты
$$
\begin{array}{rcl}
	a_0 &=& \displaystyle\frac{1}{2}\left.\left(
	\int\limits_{-1}^1
	\,dx + \int\limits_1^2
	x^2\,dx \right) = 
	\frac{1}{2}\left(
	2 +  \frac{x^3}{3}
	\right|_1^2 \right) = \frac{13}{6},												\\[12pt]
	a_n &=& \displaystyle\frac{1}{2}\left(
	\int\limits_{-1}^1
	\cos \frac{\pi nx}{2}\,dx + \int\limits_1^2
	x^2\cos \frac{\pi nx}{2}\,dx \right) ={}									\\[12pt]
	&=& \displaystyle\frac{1}{2}\left(
	\left.\frac{2}{\pi n}\sin\frac{\pi nx}{2} \right|_{-1}^1
	+\left.\frac{2}{\pi n}x^2\sin\frac{\pi nx}{2} \right|_1^2 
	+\left.\frac{8}{\pi^2 n^2}x\cos\frac{\pi nx}{2} \right|_1^2
	-\left.\frac{16}{\pi^3 n^3}\sin\frac{\pi nx}{2} \right|_1^2\right) = 	\\[12pt]
	&=& \displaystyle\left(\frac{1}{\pi n} + \frac{8}{\pi^3 n^3}\right)\sin\frac{\pi n}{2}
	+\frac{8}{\pi^2 n^2}\cos\pi n - \frac{4}{\pi^2 n^2}\cos\frac{\pi n}{2},	\\[12pt]
	b_n &=& \displaystyle\frac{1}{2}\left(
	\int\limits_{-1}^1
	\sin \frac{\pi nx}{2}\,dx + \int\limits_1^2
	x^2\sin \frac{\pi nx}{2}\,dx \right) ={}									\\[12pt]
	&=& \displaystyle\frac{1}{2}\left(
	\left.-\frac{2}{\pi n}\cos\frac{\pi nx}{2} \right|_{-1}^1
	-\left.\frac{2}{\pi n}x^2\cos\frac{\pi nx}{2} \right|_1^2 
	+\left.\frac{8}{\pi^2 n^2}x\sin\frac{\pi nx}{2} \right|_1^2
	+\left.\frac{16}{\pi^3 n^3}\cos\frac{\pi nx}{2} \right|_1^2\right) =	\\[12pt]
	&=& \displaystyle\left(\frac{8}{\pi^3 n^3}-\frac{4}{\pi n}\right)\cos\pi n + (\frac{1}{\pi n} - \frac{8}{\pi^3 n^3})\cos\frac{\pi n}{2} - \frac{4}{\pi^2n^2}\sin\frac{\pi n}{2}.
\end{array}
$$
Применив теорему Дирихле о поточечной сходимости ряда Фурье, видим, что построенный ряд Фурье сходится 
к периодическому (с периодом $T=4$) продолжению исходной функции при всех $x\ne 2+4n$, $x\ne -1+4n$, и 
$S(2+4n)=2$, $S(-1+4n)=\frac{1}{2}$ при $n=0,\pm1,\pm2,\ldots$, где $S(x)$ --- сумма ряда Ферье. 
График функции $S(x)$ имеет следующий вид
\begin{center}
	\begin{tikzpicture}
		\begin{axis}[xmin=-7, xmax=7, ymin=-1, ymax=5,
			width=0.8\textwidth,
			height=0.4\textwidth,
			axis x line=middle,
			axis y line=middle, 
			every axis x label/.style={at={(current axis.right of origin)},anchor=west},
			every inner x axis line/.append style={|-latex'},
			every inner y axis line/.append style={|-latex'},
			minor tick num=1,			
			axis equal=true,
			xlabel=$x$, 
			ylabel=$S(x)$,          
			samples=100,
			clip=true,
			]
		    \addplot[color=blue, line width=1.5pt,domain=-6:-5] {0};
		    \addplot[color=blue, line width=1.5pt,domain=-5:-3]{1};
		    \addplot[color=blue, line width=1.5pt,domain=-3:-2]{(x+4)^2)};
		    \addplot[color=blue, line width=1.5pt,domain=-2:-1] {0};
		    \addplot[color=blue, line width=1.5pt,domain=-1:1]{1};
		    \addplot[color=blue, line width=1.5pt,domain=1:2]{\x^2};
		    \addplot[color=blue, line width=1.5pt,domain=2:3] {0};
		    \addplot[color=blue, line width=1.5pt,domain=3:5]{1};
		    \addplot[color=blue, line width=1.5pt,domain=5:6]{(x-4)^2};
			\addplot[thick,dashed] coordinates {(-3,0) (-3,1)};
			\addplot[thick,dashed] coordinates {(1,0) (1,1)};
			\addplot[thick,dashed] coordinates {(5,0) (5,1)};
			\addplot[
			mark=*,
			mark options={fill=blue, draw=blue},
			only marks,
			] coordinates {(-5, 0.5) (-2, 2) (-1, 0.5) (2, 2) (3, 0.5)};
		\end{axis}
	\end{tikzpicture}
\end{center}
\noindent
\textbf{Ответ:}


\[
\begin{split}
	&f(x)=\frac{13}{12}+\sum_{n=1}^\infty [\left( \left(\frac{1}{\pi n} + \frac{8}{\pi^3 n^3} \right)\sin\frac{\pi n}{2} + \frac{8}{\pi^2 n^2}\cos\pi n -\frac{4}{\pi^2 n^2}\cos\frac{\pi n}{2} \right) \cos\frac{\pi n x}{2} + \\
	&+ \left( \left( \frac{8}{\pi^3 n^3} - \frac{4}{\pi n} \right)\cos\pi n + \left( \frac{1}{\pi n} - \frac{8}{\pi^3 n^3}\right)\cos\frac{\pi n}{2} - \frac{4}{\pi^2 n^2}\sin\frac{\pi n}{2}\right)\sin\frac{\pi n x}{2}], x\ne 2+4n, x\ne -1+4n; \\
	&S(2+4n)=2, \text{ при } n\in\mathbb{Z}, \\
	&S(-1+4n)=\frac{1}{2}, \text{ при } n\in\mathbb{Z}.
\end{split}
\]


% ---------------------------- Problem 2----------------------------------
\subsubsection*{\center Задача № 2.}
{\bf Условие.~}
Для заданной графически функции $y(x)$ построить ряд Фурье в комплексной форме, изобразить график суммы построенного ряда

%График
\begin{center}
	\begin{tikzpicture}[
		declare function={
			func(\x)=
			and(\x >= 0, \x <= 2) * 0.5 * (\x) + 
			and(\x >  2, \x <= 3) * 0.0;
		}
		]
		\begin{axis}[
			axis x line=middle, axis y line=middle,
			axis equal,	
			ymin=-1.1, ymax=1.1, ytick={-1,...,1}, ylabel=$y$,
			xmin=-1.1, xmax=5, xtick={-1,...,4}, xlabel=$x$,
			domain=-0.0:3.0,samples=600, % added
			]
			
			\addplot [domain=0:2,blue,line width=2pt] {0.5*\x};
			\addplot [domain=2:3,blue,line width=2pt] {0.0};
			\addplot [dashed, black] coordinates {(2,0)(2,1)};		
		\end{axis}
	\end{tikzpicture}
\end{center}

\noindent
\textbf{Решение.}\\

\noindent
Ряд Фурье в комплексной форме имеет следующий вид
\[
f(x) = \sum_{n=-\infty}^\infty c_n e^{i\omega nx},\quad c_n=\frac{1}{T}\int\limits_a^b f(x) e^{-i\omega nx}dx,\,\omega=\frac{2\pi}{T}.
\]
В нашем примере $ a=0,b=3,T=3,\omega=2\pi/3$, 
найдем коэффицинеты $c_n,\,n=0,\pm1,\pm2,\ldots$
где $\omega=2\pi/T,\,T=3.$
$$
\begin{array}{rcl}
	c_0 &=&\displaystyle\frac{1}{3} \int\limits_0^3 f(x)dx=\frac{a_0}{2}=\frac{1}{3}\int\limits_0^2 \frac{x}{2}dx=\frac{1}{3},\\[12pt]
	c_n &=&\displaystyle\frac{1}{3}
	\int\limits_0^2
	\frac{x}{2}e^{-i \frac{2}{3} \pi nx}dx ={}\\[12pt]
	&=&\displaystyle\frac{1}{6} \left(\left.\frac{3xi}{2\pi n} e^{-i\frac{2}{3}\pi nx} +
	\frac{9}{4\pi^2 n^2} e^{-i \frac{2}{3} \pi nx} \right) \right|_0^2 = \\[12pt]
	&=&\displaystyle \frac{i}{2\pi n} e^{-i \frac{4}{3} \pi n} + \frac{3}{8 \pi^2 n^2} e^{-i \frac{4}{3} \pi n } - \frac{3}{8 \pi^2 n^2} = \frac{1}{2 \pi n} \left( \sin\frac{4 \pi n}{3} + \frac{3(\cos\frac{4\pi n}{3} - 1)}{4 \pi n} \right) + \\
	&&\displaystyle+ \frac{i}{2\pi n} \left( \cos\frac{4\pi n}{3} - \frac{3\sin\frac{4\pi n}{3}}{4\pi n} \right).
\end{array} 
$$
\noindent
Применив теорему Дирихле о поточечной сходимости ряда Фурье, видим, что построенный ряд Фурье сходится 
к периодическому (с периодом $T=3$) продолжению исходной функции при всех $x\ne 2+3n$, и $S(2+3n)=1/2$ при 
$n=0,\pm1,\pm2,\ldots$, где $S(x)$ --- сумма ряда Фурье. График функции $S(x)$ имеет вид
\begin{center}
	\begin{tikzpicture}
		\begin{axis}[xmin=-6, xmax=6, ymin=-1, ymax=2,
			width=0.8\textwidth,
			height=0.4\textwidth,
			axis x line=middle,
			axis y line=middle, 
			every axis x label/.style={at={(current axis.right of origin)},anchor=west},
			every inner x axis line/.append style={|-latex'},
			every inner y axis line/.append style={|-latex'},
			minor tick num=1,			
			axis equal=true,
			xlabel=$x$, 
			ylabel=$S(x)$,          
			samples=100,
			clip=true,
			]
			\addplot[color=blue, line width=1.5pt,domain=-6:-4] {(x+6)/2};
			\addplot[color=blue, line width=1.5pt,domain=-4:-3]{0};
			\addplot[color=blue, line width=1.5pt,domain=-3:-1] {(x+3)/2};
			\addplot[color=blue, line width=1.5pt,domain=-1:0]{0};
			\addplot[color=blue, line width=1.5pt,domain=0:2] {x/2};
			\addplot[color=blue, line width=1.5pt,domain=2:3]{0};
			\addplot[color=blue, line width=1.5pt,domain=3:5] {(x-3)/2};
			\addplot[color=blue, line width=1.5pt,domain=5:6]{0};
			\addplot[
			mark=*,
			mark options={fill=blue, draw=black},
			only marks,
			] coordinates {(-4, 0.5) (-1, 0.5) (2, 0.5) (5, 0.5)};
		\end{axis}
	\end{tikzpicture}
\end{center}

\noindent
\textbf{Ответ:}
\[
\begin{split}
	&f(x)=\sum_{n=-\infty}^\infty\left[ \frac{1}{2\pi n} \left( \sin\frac{4 \pi n}{3} + \frac{3(\cos\frac{4\pi n}{3} - 1)}{4\pi n} \right) + \frac{i}{2\pi n} \left( \cos\frac{4\pi n}{3} - \frac{3\sin\frac{4\pi n}{3}}{4\pi n} \right) \right] e^{\tfrac{i2\pi nx}{3}},~ x\ne 2+3n; \\
	&S(2+3n)=\frac{1}{2},\quad\text{при}~n\in\mathbb{Z}.
\end{split}
\]


% ---------------------------- Problem 3----------------------------------
\subsubsection*{\center Задача № 3.}
{\bf Условие.~}\\
Найти резольвенту для интегрального уравнения Вольтерры со следующим ядром 
$$K(x,t)=(x-t)2^{(\sin{x}-\sin{t})},\,\lambda=4$$

\noindent
{\bf Решение.~}\\
\noindent
Из рекурентных соотношений получаем
$$
\begin{array}{rcl}
	K_1(x,t)&=&\displaystyle (x-t)2^{(\sin x - \sin t)}, \\[12pt]
	K_2(x,t)&=&\displaystyle\int\limits_t^x K(x,s)K_1(s,t)ds = \int\limits_t^x (x-s)2^{(\sin x - \sin s)} (s-t)2^{(\sin s - \sin t)} ds = \\[12pt]
	&=&\displaystyle 2^{(\sin x - \sin t)}\int\limits_t^x (xs - xt - s^{2} +st)ds = 2^{(\sin x -\sin t)} \frac{(x-t)^{3}}{3!},\\[12pt]
	K_3(x,t)&=&\displaystyle\int\limits_t^x K(x,s)K_2(s,t)ds = \int\limits_t^x (x-s)2^{(\sin x - \sin s)}2^{(\sin s - \sin t)} \frac{(s-t)^{3}}{6}ds = \\[12pt]
	&=&\displaystyle 2^{(\sin x - \sin t)} \frac{1}{6} \int\limits_t^x (xs^{3}-3s^{2}tx+3sxt^{2} -xt^{3} - s^{4} +3ts^{3} -3s^{2}t^{2}+st^{3})ds =  \\[12pt]
	&=&\displaystyle 2^{(\sin x -\sin t)} \frac{(x-t)^{5}}{5!},  \\[12pt]
	K_j(x,t)&=&\displaystyle 2^{(\sin x -\sin t)} \frac{(x-t)^{2j-1}}{\left(2j-1\right)!}  \!,\quad j=\mathbb{N}.
\end{array}
$$
Подставляя это выражение для итерированных ядер, найдем резольвенту
$$ 
R(x,t,\lambda)= 2^{(\sin x - \sin t)} \sum_{j=1}^\infty \frac{4^{j-1}}{(2j-1)!}\left( x-t \right)^{2j-1}\!\!\!\!\!\!\!\!,
\quad j=1,2,\ldots
$$
